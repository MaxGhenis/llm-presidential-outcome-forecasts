\section{Technical Implementation}\label{sec:technical}

The implementation consists of several key components designed to ensure reproducible, robust analysis of model predictions.

\subsection{API Integration}

A modular system was developed to interact with multiple model APIs:

\begin{itemize}
    \item \textbf{Unified Interface:} Common wrapper for GPT-4, GPT-4-mini, and Grok APIs
    \item \textbf{Rate Limiting:} Automatic handling of API rate limits and quotas
    \item \textbf{Error Handling:} Robust recovery from API failures and timeouts
    \item \textbf{Response Validation:} Immediate validation of API responses
\end{itemize}

The system uses the $n$ keyword in the API calls to request multiple responses without duplicating input tokens.

\subsection{Response Processing}

Given the narrative format of responses, specialized processing was required to extract consistent numerical predictions:

\begin{itemize}
    \item \textbf{Number Extraction:} Robust regular expressions for various formats:
    \begin{itemize}
        \item Direct numerical mentions (e.g., "8.5")
        \item Percentage representations (e.g., "8.5\%")
        \item Mixed format numbers (e.g., "8.5 percent")
    \end{itemize}
    
    \item \textbf{Validation Rules:} Multi-stage validation process:
    \begin{itemize}
        \item Format validation (correct numerical format)
        \item Range validation (within reasonable bounds)
        \item Consistency validation (matches narrative context)
    \end{itemize}
    
    \item \textbf{Statistical Processing:} Tools for aggregating results:
    \begin{itemize}
        \item Outlier detection and handling
        \item Distribution analysis
        \item Confidence interval calculation
    \end{itemize}
\end{itemize}

\subsection{Analysis Pipeline}

The analysis pipeline processes raw model outputs through several stages:

\begin{enumerate}
    \item \textbf{Data Collection:}
    \begin{itemize}
        \item Parallel API calls for efficiency
        \item Automatic retry on failure
        \item Response logging and backup
    \end{itemize}
    
    \item \textbf{Data Cleaning:}
    \begin{itemize}
        \item Number extraction and validation
        \item Outlier detection and handling
        \item Format standardization
    \end{itemize}
    
    \item \textbf{Statistical Analysis:}
    \begin{itemize}
        \item Effect size calculation
        \item Variance analysis
        \item Model comparison statistics
    \end{itemize}
    
    \item \textbf{Visualization:}
    \begin{itemize}
        \item Distribution plots
        \item Effect size comparisons
        \item Time series analysis
    \end{itemize}
\end{enumerate}

\subsection{Reproducibility}

The implementation emphasizes reproducibility through:

\begin{itemize}
    \item \textbf{Version Control:} All code and prompts in Git repository
    \item \textbf{Environment Management:} Dependencies specified in requirements.txt
    \item \textbf{Data Storage:} Raw API responses preserved
    \item \textbf{Random Seed Control:} Fixed seeds for reproducible sampling
\end{itemize}

The complete implementation is available at \url{https://github.com/MaxGhenis/llm-presidential-outcome-forecasts}.